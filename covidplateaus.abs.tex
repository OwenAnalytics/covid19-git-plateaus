The COVID-19 pandemic has caused more than XXXX reported deaths
globally, of which more than YYY have been reported
in the United States as of XXXX. Public health interventions
have had significant impacts in reducing transmission and in
averting even more deaths. Nonetheless, in many jurisdictions
the decline of cases and fatalities
after apparent epidemic peaks has not been rapid.  Instead, the asymmetric
decline in cases appears, in most cases, to be consistent
with plateau- or shoulder-like phenomena -- a qualitative
observation reinforced by a symmetry analysis
of US state-level fatality data.  Here we explore a model of fatality-driven
awareness in which individual protective measures increase
with death rates.  In this model, epidemic dynamics
can be characterized by plateaus, shoulders,
and lag-driven oscillations after exponential rises
at the outset of disease dynamics. 
We find that model-predicted outcomes
are consistent with asymmetric shape of observed peaks. 
Yet, in contrast
to model predictions, we find that population-level
estimates of mobility usually increased \emph{before} 
peak levels of fatalities.  We show that
incorporating fatigue and
and long-term behavior change can reconcile the apparent
premature relaxation of mobility reductions and dictate whether
post-peak dynamics leads to a resurgence of cases or epidemic declines.
