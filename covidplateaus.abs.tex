The COVID-19 pandemic has caused more than 200,000 reported deaths
globally, of which more than 50,000 have been reported
in the United States. Public health interventions
have had significant impacts in reducing transmission and in
averting even more deaths. Nonetheless, in many jurisdictions
(both at national and local levels) the decline of cases and fatalities
after apparent epidemic peaks has not been rapid.  Instead, the asymmetric
decline in cases appears, in some cases, to be consistent
with plateau- or shoulder-like phenomena.  
Here we explore a model of fatality-driven
awareness in which individual protective measures increase
with death rates.  In this model, epidemic dynamics
can be characterized by plateaus, shoulders,
and lag-driven oscillations after exponential rises
at the outset of disease dynamics. We also show that
incorporating long-term awareness can avoid peak resurgence and accelerate
epidemic decline.  We suggest that awareness of the severity of the short- and long-term epidemic is likely to play a critical
role in disease dynamics, beyond that imposed by intervention-driven policies.
