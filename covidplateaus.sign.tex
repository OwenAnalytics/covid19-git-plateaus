\begin{boxit}
\noindent
\textbf{Significance statement:}
In contrast to predictions of conventional epidemic models,
COVID-19 cases and fatalities have asymmetric shapes at both national
and sub-national scales such that cases and fatalities 
rise rapidly and decay slowly.  This manuscript evaluates how awareness-driven
behavior modulates the shape of epidemics.
We find that short-term awareness of fatalities leads 
to emergent plateaus, persistent shoulder-like dynamics, and lag-driven
oscillations in a SEIR-like model.  
Hence, new cases and fatalities persist at 
a near-constant rate given an almost entirely susceptible population.
We also find that incorporating long-term awareness 
accelerates epidemic decline, leading to a switch from plateaus back to peak-like dynamics. 
These findings suggest the need to incorporate feedback
between outbreak and behavior in forecasting models and in evaluating public
health campaigns to control epidemic spread.
\end{boxit}
