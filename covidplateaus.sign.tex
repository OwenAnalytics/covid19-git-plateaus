\boxit{
\noindent
\textbf{Significance statement:}\\
In contrast to predictions of conventional epidemic models, COVID-19 outbreak time series have asymmetric shapes, with cases and fatalities declining much more slowly than they rose. 
Here, we investigate how awareness-driven behavior modulates epidemic shape. 
We find that short-term awareness of fatalities leads to emergent plateaus, persistent shoulder-like dynamics, and lag-driven oscillations in a SEIR-like model; consistent with observed disease time series from the US states. 
However, a joint analysis of fatalities and mobility suggest that populations relaxed mobility restrictions prior to fatality peaks, in contrast to model predictions. 
We show that incorporating fatigue and long-term behavior change can explain this phenomenon, and shed light on when post-peak dynamics are likely to lead to a resurgence of cases or to  sustained declines. These findings suggest the need to incorporate behavior-driven feedback in epidemic models and in public health campaigns to control COVID-19 spread.
}
